% \typeout{Academic Literature Management System (Enhanced Version with Tag Support) loaded successfully!}

% ========================================================================
% 学术文献管理系统定义文件 (literature-system.tex) - 增强版本
% ========================================================================
%
% 【系统目标】
% 本系统旨在为学术研究者提供一个完整的文献管理解决方案,通过LaTeX自动化处理
% 文献的录入、分类、排序和展示,提高学术写作效率和文献综述质量。
%
% 【核心理念】
% - 结构化存储:将文献信息以结构化方式存储,便于批量处理和自动化操作
% - 智能排序:支持多字段组合排序,满足不同学术场景的展示需求
% - 标签分类:通过标签系统实现文献的分类管理和快速检索
% - 格式统一:提供多种预定义格式,确保文献展示的一致性和专业性
% - 扩展性强:模块化设计,便于根据具体需求进行功能扩展
%
% 【使用场景】
% 1. 学位论文文献综述章节
% 2. 学术论文相关工作部分
% 3. 研究报告参考文献管理
% 4. 课程作业文献整理
% 5. 研究项目文献数据库构建
%
% 【使用方法】
% 在主LaTeX文档中添加:% \typeout{Academic Literature Management System (Enhanced Version with Tag Support) loaded successfully!}

% ========================================================================
% 学术文献管理系统定义文件 (literature-system.tex) - 增强版本
% ========================================================================
%
% 【系统目标】
% 本系统旨在为学术研究者提供一个完整的文献管理解决方案,通过LaTeX自动化处理
% 文献的录入、分类、排序和展示,提高学术写作效率和文献综述质量。
%
% 【核心理念】
% - 结构化存储:将文献信息以结构化方式存储,便于批量处理和自动化操作
% - 智能排序:支持多字段组合排序,满足不同学术场景的展示需求
% - 标签分类:通过标签系统实现文献的分类管理和快速检索
% - 格式统一:提供多种预定义格式,确保文献展示的一致性和专业性
% - 扩展性强:模块化设计,便于根据具体需求进行功能扩展
%
% 【使用场景】
% 1. 学位论文文献综述章节
% 2. 学术论文相关工作部分
% 3. 研究报告参考文献管理
% 4. 课程作业文献整理
% 5. 研究项目文献数据库构建
%
% 【使用方法】
% 在主LaTeX文档中添加:% \typeout{Academic Literature Management System (Enhanced Version with Tag Support) loaded successfully!}

% ========================================================================
% 学术文献管理系统定义文件 (literature-system.tex) - 增强版本
% ========================================================================
%
% 【系统目标】
% 本系统旨在为学术研究者提供一个完整的文献管理解决方案,通过LaTeX自动化处理
% 文献的录入、分类、排序和展示,提高学术写作效率和文献综述质量。
%
% 【核心理念】
% - 结构化存储:将文献信息以结构化方式存储,便于批量处理和自动化操作
% - 智能排序:支持多字段组合排序,满足不同学术场景的展示需求
% - 标签分类:通过标签系统实现文献的分类管理和快速检索
% - 格式统一:提供多种预定义格式,确保文献展示的一致性和专业性
% - 扩展性强:模块化设计,便于根据具体需求进行功能扩展
%
% 【使用场景】
% 1. 学位论文文献综述章节
% 2. 学术论文相关工作部分
% 3. 研究报告参考文献管理
% 4. 课程作业文献整理
% 5. 研究项目文献数据库构建
%
% 【使用方法】
% 在主LaTeX文档中添加:% \typeout{Academic Literature Management System (Enhanced Version with Tag Support) loaded successfully!}

% ========================================================================
% 学术文献管理系统定义文件 (literature-system.tex) - 增强版本
% ========================================================================
%
% 【系统目标】
% 本系统旨在为学术研究者提供一个完整的文献管理解决方案,通过LaTeX自动化处理
% 文献的录入、分类、排序和展示,提高学术写作效率和文献综述质量。
%
% 【核心理念】
% - 结构化存储:将文献信息以结构化方式存储,便于批量处理和自动化操作
% - 智能排序:支持多字段组合排序,满足不同学术场景的展示需求
% - 标签分类:通过标签系统实现文献的分类管理和快速检索
% - 格式统一:提供多种预定义格式,确保文献展示的一致性和专业性
% - 扩展性强:模块化设计,便于根据具体需求进行功能扩展
%
% 【使用场景】
% 1. 学位论文文献综述章节
% 2. 学术论文相关工作部分
% 3. 研究报告参考文献管理
% 4. 课程作业文献整理
% 5. 研究项目文献数据库构建
%
% 【使用方法】
% 在主LaTeX文档中添加:\input{literature-system.tex}
%
% 【系统架构】
% 1. 数据存储层:基于datatool包的数据库系统
% 2. 数据处理层:排序、筛选、统计功能
% 3. 显示控制层:多种格式化显示模板
% 4. 用户接口层:简化的命令接口
%
% 【技术依赖】
% - datatool: 核心数据处理包
% - longtable: 长表格支持(跨页显示)
% - array: 表格格式增强
% - xcolor: 颜色支持(标签高亮)
% - xstring: 字符串处理(标签解析)
%
% ========================================================================

% 必需的包(如果主文档没有加载的话)
\RequirePackage{datatool}
\RequirePackage{longtable}
\RequirePackage{array}
\RequirePackage{xcolor}

% ========================================================================
% 1. 数据库创建
% ========================================================================

% 创建文献数据库
\DTLnewdb{literature}

% ========================================================================
% 2. 数据添加接口(支持标签)
% ========================================================================

% 添加文献数据的主要命令(支持标签参数)
% 用法:\addliterature{标题}{作者}{年份}{洞察}{标签}
\newcommand{\addliterature}[5]{%
\DTLnewrow{literature}%
\DTLnewdbentry{literature}{title}{#1}%
\DTLnewdbentry{literature}{authors}{#2}%
\DTLnewdbentry{literature}{year}{#3}%
\DTLnewdbentry{literature}{insight}{#4}%
\DTLnewdbentry{literature}{tag}{#5}%
}

% 兼容旧版本的四参数命令(自动设置空标签)
\newcommand{\addliteratureold}[4]{%
\addliterature{#1}{#2}{#3}{#4}{}%
}

% ========================================================================
% 3. 排序功能
% ========================================================================

% 按年份排序(数值排序,从早到晚)
\newcommand{\sortbyyear}{%
\DTLsort{year}{literature}%
}

% 按标题排序(字母排序)
\newcommand{\sortbytitle}{%
\DTLsort{title}{literature}%
}

% 按作者排序(字母排序)
\newcommand{\sortbyauthors}{%
\DTLsort{authors}{literature}%
}

% 按标签排序(字母排序)
\newcommand{\sortbytag}{%
\DTLsort{tag}{literature}%
}

% 自定义排序命令
\newcommand{\sortby}[1]{%
\DTLsort{#1}{literature}%
}

% 复合排序(先按年份,再按标题)
\newcommand{\sortbyyeartitle}{%
\DTLsort{year,title}{literature}%
}

% 复合排序(先按标签,再按年份)
\newcommand{\sortbytagyear}{%
\DTLsort{tag,year}{literature}%
}

% ========================================================================
% 4. 空字段处理函数
% ========================================================================

% 处理空insight字段的显示函数
\makeatletter
\newcommand{\displayinsight}[1]{%
\def\@tempa{#1}%
\ifx\@tempa\@empty
\emptyset
\else
#1%
\fi
}

% 处理空tag字段的显示函数
\newcommand{\displaytag}[1]{%
\def\@tempa{#1}%
\ifx\@tempa\@empty
\textit{无标签}
\else
\textcolor{blue}{\texttt{#1}}%
\fi
}
\makeatother

% ========================================================================
% 5. 显示格式定义(支持标签显示)
% ========================================================================

% 文献编号计数器
\newcounter{papercount}

% 重置文献编号
\newcommand{\resetpapercount}{%
\setcounter{papercount}{0}%
}

% 单个文献条目显示命令(详细格式)
\newcommand{\showpaperentry}[5]{%
\stepcounter{papercount}%
\textbf{Paper \arabic{papercount}} & \\
\hline
Title & #1 \\
\hline
Authors & #2 \\
\hline
Year & #3 \\
\hline
Tag & \displaytag{#5} \\
\hline
Insight & \displayinsight{#4} \\
\hline\hline
}

% 简化版文献条目显示(只显示基本信息和标签)
\newcommand{\showpaperbasic}[5]{%
\stepcounter{papercount}%
\textbf{\arabic{papercount}.} & \textbf{#1} (#3) [\displaytag{#5}] \\
& \textit{#2} \\
\hline
}

% 带insight的简化版显示(包含标签)
\newcommand{\showpaperbasicwithinsight}[5]{%
\stepcounter{papercount}%
\textbf{\arabic{papercount}.} & \textbf{#1} (#3) [\displaytag{#5}] \\
& \textit{#2} \\
& \displayinsight{#4} \\
\hline
}

% ========================================================================
% 6. 显示环境定义
% ========================================================================

% 完整文献综述表格环境
\newenvironment{literaturereview}
{\begin{longtable}{|p{0.15\textwidth}|p{0.8\textwidth}|}
\hline
\textbf{Field} & \textbf{Content} \\
\hline
\endhead}
{\end{longtable}}

% 简化版文献列表环境
\newenvironment{literaturelist}
{\begin{longtable}{|p{0.05\textwidth}|p{0.9\textwidth}|}
\hline
\textbf{No.} & \textbf{Literature} \\
\hline
\endhead}
{\end{longtable}}

% ========================================================================
% 7. 批量显示命令(支持标签)
% ========================================================================

% 显示完整文献综述
\newcommand{\showfullreview}{%
\resetpapercount%
\begin{literaturereview}
\DTLforeach{literature}{\thetitle=title,\theauthors=authors,\theyear=year,\theinsight=insight,\thetag=tag}{%
\showpaperentry{\thetitle}{\theauthors}{\theyear}{\theinsight}{\thetag}%
}
\end{literaturereview}
}

% 显示简化版文献列表
\newcommand{\showbasiclist}{%
\resetpapercount%
\begin{literaturelist}
\DTLforeach{literature}{\thetitle=title,\theauthors=authors,\theyear=year,\theinsight=insight,\thetag=tag}{%
\showpaperbasic{\thetitle}{\theauthors}{\theyear}{\theinsight}{\thetag}%
}
\end{literaturelist}
}

% 显示带insight的简化版文献列表
\newcommand{\showbasiclistwithinsight}{%
\resetpapercount%
\begin{literaturelist}
\DTLforeach{literature}{\thetitle=title,\theauthors=authors,\theyear=year,\theinsight=insight,\thetag=tag}{%
\showpaperbasicwithinsight{\thetitle}{\theauthors}{\theyear}{\theinsight}{\thetag}%
}
\end{literaturelist}
}

% ========================================================================
% 8. 标签筛选显示功能
% ========================================================================

% 按指定标签显示完整文献综述
\newcommand{\showfullreviewbytag}[1]{%
\resetpapercount%
\begin{literaturereview}
\DTLforeach[\DTLisSubString{\thetag}{#1}]{literature} {\thetitle=title,\theauthors=authors,\theyear=year,\theinsight=insight,\thetag=tag}{%
\showpaperentry{\thetitle}{\theauthors}{\theyear}{\theinsight}{\thetag}%
}
\end{literaturereview}
}

% 按指定标签显示简化版文献列表
\newcommand{\showbasiclistbytag}[1]{%
\resetpapercount%
\begin{literaturelist}
\DTLforeach[\DTLisSubString{\thetag}{#1}]{literature} {\thetitle=title,\theauthors=authors,\theyear=year,\theinsight=insight,\thetag=tag}{%
\showpaperbasic{\thetitle}{\theauthors}{\theyear}{\theinsight}{\thetag}%
}
\end{literaturelist}
}

% 按指定标签显示带insight的简化版文献列表
\newcommand{\showbasiclistwithinsightbytag}[1]{%
\resetpapercount%
\begin{literaturelist}
\DTLforeach[\DTLisSubString{\thetag}{#1}]{literature} {\thetitle=title,\theauthors=authors,\theyear=year,\theinsight=insight,\thetag=tag}{%
\showpaperbasicwithinsight{\thetitle}{\theauthors}{\theyear}{\theinsight}{\thetag}%
}
\end{literaturelist}
}
 
% 需要添加的包
\RequirePackage{xstring}

% 智能标签解析与显示系统
\makeatletter
\newcommand{\showavailabletags}{%
  \textbf{Available Tags:}
  \def\@alltagslist{}%
  \DTLforeach{literature}{\thetag=tag}{%
      \ifx\thetag\empty
      \else
          \@processtags{\thetag}%
      \fi
  }%
  \@showtaglist%
  \\
}

% 处理逗号分隔标签的函数
\def\@processtags#1{%
  \StrSubstitute{#1}{ }{}[\@cleanedtags]%
  \@for\@singletag:=\@cleanedtags\do{%
      \ifx\@singletag\empty
      \else
          \@addtolistifnotexist{\@singletag}%
      \fi
  }%
}

% 添加标签到列表函数
\def\@addtolistifnotexist#1{%
  \@tempswafalse
  \ifx\@alltagslist\empty
      \edef\@alltagslist{#1}%
  \else
      \@checktagexists{#1}%
      \if@tempswa\else
          \edef\@alltagslist{\@alltagslist,#1}%
      \fi
  \fi
}

% 检查标签是否存在函数
\def\@checktagexists#1{%
  \@tempswafalse
  \@for\@existingtag:=\@alltagslist\do{%
      \StrCompare{#1}{\@existingtag}[\@result]%
      \ifnum\@result=0
          \@tempswatrue
      \fi
  }%
}

% 显示标签列表函数
\def\@showtaglist{%
  \ifx\@alltagslist\empty
      \textit{No tags found}
  \else
      \@for\@singletag:=\@alltagslist\do{%
          \ifx\@singletag\empty\else
              \displaytag{\@singletag}\space%
          \fi
      }%
  \fi
}
\makeatother

% ========================================================================
% 9. 统计和查询功能(支持标签统计)
% ========================================================================

% 获取文献总数
\newcommand{\getpapercount}{%
\DTLrowcount{literature}%
}

% 获取指定标签的文献数量
\newcommand{\getpapercountbytag}[1]{%
\DTLcondall{literature}{\DTLiseq{\thetag}{#1}}{\thetag=tag}%
}

% 显示统计信息
\newcommand{\showstats}{%
\begin{itemize}
\item Total papers: \getpapercount
\item Database: literature
\item Available fields: title, authors, year, insight, tag
\item \showavailabletags
\end{itemize}
}

% 按年份范围筛选显示(框架)
\newcommand{\showyearrange}[2]{%
\textbf{Papers from #1 to #2:}
}

% 按标签和年份范围组合筛选
\newcommand{\showbytagyearrange}[3]{%
\textbf{Papers with tag "#1" from #2 to #3:}
}

% ========================================================================
% 10. 实用工具命令
% ========================================================================

% 清空数据库
\newcommand{\clearliterature}{%
\DTLcleardb{literature}%
}

% 保存数据库到文件(接口定义)
\newcommand{\saveliterature}[1]{%
\PackageWarning{literature-system}{Save function not implemented yet}%
}

% 从文件加载数据库
\newcommand{\loadliterature}[1]{%
\PackageWarning{literature-system}{Load function not implemented yet}%
}

% ========================================================================
% 系统信息
% ========================================================================

\newcommand{\systeminfo}{%
\begin{center}
\fbox{\begin{minipage}{0.8\textwidth}
\textbf{Literature Auto-Sort System v2.0 (With Tags)}\\
\textit{Powered by datatool package}\\
Features: Auto-sorting, Multi-format display, Statistics, Tag filtering, Empty insight as $\emptyset$\\
File: literature-system.tex (Tag-enabled Version)
\end{minipage}}
\end{center}
}

% ========================================================================
% 文件结束标记
% ========================================================================
\typeout{Literature Auto-Sort System (Tag-enabled Version) loaded successfully!}
%
% 【系统架构】
% 1. 数据存储层:基于datatool包的数据库系统
% 2. 数据处理层:排序、筛选、统计功能
% 3. 显示控制层:多种格式化显示模板
% 4. 用户接口层:简化的命令接口
%
% 【技术依赖】
% - datatool: 核心数据处理包
% - longtable: 长表格支持(跨页显示)
% - array: 表格格式增强
% - xcolor: 颜色支持(标签高亮)
% - xstring: 字符串处理(标签解析)
%
% ========================================================================

% 必需的包(如果主文档没有加载的话)
\RequirePackage{datatool}
\RequirePackage{longtable}
\RequirePackage{array}
\RequirePackage{xcolor}

% ========================================================================
% 1. 数据库创建
% ========================================================================

% 创建文献数据库
\DTLnewdb{literature}

% ========================================================================
% 2. 数据添加接口(支持标签)
% ========================================================================

% 添加文献数据的主要命令(支持标签参数)
% 用法:\addliterature{标题}{作者}{年份}{洞察}{标签}
\newcommand{\addliterature}[5]{%
\DTLnewrow{literature}%
\DTLnewdbentry{literature}{title}{#1}%
\DTLnewdbentry{literature}{authors}{#2}%
\DTLnewdbentry{literature}{year}{#3}%
\DTLnewdbentry{literature}{insight}{#4}%
\DTLnewdbentry{literature}{tag}{#5}%
}

% 兼容旧版本的四参数命令(自动设置空标签)
\newcommand{\addliteratureold}[4]{%
\addliterature{#1}{#2}{#3}{#4}{}%
}

% ========================================================================
% 3. 排序功能
% ========================================================================

% 按年份排序(数值排序,从早到晚)
\newcommand{\sortbyyear}{%
\DTLsort{year}{literature}%
}

% 按标题排序(字母排序)
\newcommand{\sortbytitle}{%
\DTLsort{title}{literature}%
}

% 按作者排序(字母排序)
\newcommand{\sortbyauthors}{%
\DTLsort{authors}{literature}%
}

% 按标签排序(字母排序)
\newcommand{\sortbytag}{%
\DTLsort{tag}{literature}%
}

% 自定义排序命令
\newcommand{\sortby}[1]{%
\DTLsort{#1}{literature}%
}

% 复合排序(先按年份,再按标题)
\newcommand{\sortbyyeartitle}{%
\DTLsort{year,title}{literature}%
}

% 复合排序(先按标签,再按年份)
\newcommand{\sortbytagyear}{%
\DTLsort{tag,year}{literature}%
}

% ========================================================================
% 4. 空字段处理函数
% ========================================================================

% 处理空insight字段的显示函数
\makeatletter
\newcommand{\displayinsight}[1]{%
\def\@tempa{#1}%
\ifx\@tempa\@empty
\emptyset
\else
#1%
\fi
}

% 处理空tag字段的显示函数
\newcommand{\displaytag}[1]{%
\def\@tempa{#1}%
\ifx\@tempa\@empty
\textit{无标签}
\else
\textcolor{blue}{\texttt{#1}}%
\fi
}
\makeatother

% ========================================================================
% 5. 显示格式定义(支持标签显示)
% ========================================================================

% 文献编号计数器
\newcounter{papercount}

% 重置文献编号
\newcommand{\resetpapercount}{%
\setcounter{papercount}{0}%
}

% 单个文献条目显示命令(详细格式)
\newcommand{\showpaperentry}[5]{%
\stepcounter{papercount}%
\textbf{Paper \arabic{papercount}} & \\
\hline
Title & #1 \\
\hline
Authors & #2 \\
\hline
Year & #3 \\
\hline
Tag & \displaytag{#5} \\
\hline
Insight & \displayinsight{#4} \\
\hline\hline
}

% 简化版文献条目显示(只显示基本信息和标签)
\newcommand{\showpaperbasic}[5]{%
\stepcounter{papercount}%
\textbf{\arabic{papercount}.} & \textbf{#1} (#3) [\displaytag{#5}] \\
& \textit{#2} \\
\hline
}

% 带insight的简化版显示(包含标签)
\newcommand{\showpaperbasicwithinsight}[5]{%
\stepcounter{papercount}%
\textbf{\arabic{papercount}.} & \textbf{#1} (#3) [\displaytag{#5}] \\
& \textit{#2} \\
& \displayinsight{#4} \\
\hline
}

% ========================================================================
% 6. 显示环境定义
% ========================================================================

% 完整文献综述表格环境
\newenvironment{literaturereview}
{\begin{longtable}{|p{0.15\textwidth}|p{0.8\textwidth}|}
\hline
\textbf{Field} & \textbf{Content} \\
\hline
\endhead}
{\end{longtable}}

% 简化版文献列表环境
\newenvironment{literaturelist}
{\begin{longtable}{|p{0.05\textwidth}|p{0.9\textwidth}|}
\hline
\textbf{No.} & \textbf{Literature} \\
\hline
\endhead}
{\end{longtable}}

% ========================================================================
% 7. 批量显示命令(支持标签)
% ========================================================================

% 显示完整文献综述
\newcommand{\showfullreview}{%
\resetpapercount%
\begin{literaturereview}
\DTLforeach{literature}{\thetitle=title,\theauthors=authors,\theyear=year,\theinsight=insight,\thetag=tag}{%
\showpaperentry{\thetitle}{\theauthors}{\theyear}{\theinsight}{\thetag}%
}
\end{literaturereview}
}

% 显示简化版文献列表
\newcommand{\showbasiclist}{%
\resetpapercount%
\begin{literaturelist}
\DTLforeach{literature}{\thetitle=title,\theauthors=authors,\theyear=year,\theinsight=insight,\thetag=tag}{%
\showpaperbasic{\thetitle}{\theauthors}{\theyear}{\theinsight}{\thetag}%
}
\end{literaturelist}
}

% 显示带insight的简化版文献列表
\newcommand{\showbasiclistwithinsight}{%
\resetpapercount%
\begin{literaturelist}
\DTLforeach{literature}{\thetitle=title,\theauthors=authors,\theyear=year,\theinsight=insight,\thetag=tag}{%
\showpaperbasicwithinsight{\thetitle}{\theauthors}{\theyear}{\theinsight}{\thetag}%
}
\end{literaturelist}
}

% ========================================================================
% 8. 标签筛选显示功能
% ========================================================================

% 按指定标签显示完整文献综述
\newcommand{\showfullreviewbytag}[1]{%
\resetpapercount%
\begin{literaturereview}
\DTLforeach[\DTLisSubString{\thetag}{#1}]{literature} {\thetitle=title,\theauthors=authors,\theyear=year,\theinsight=insight,\thetag=tag}{%
\showpaperentry{\thetitle}{\theauthors}{\theyear}{\theinsight}{\thetag}%
}
\end{literaturereview}
}

% 按指定标签显示简化版文献列表
\newcommand{\showbasiclistbytag}[1]{%
\resetpapercount%
\begin{literaturelist}
\DTLforeach[\DTLisSubString{\thetag}{#1}]{literature} {\thetitle=title,\theauthors=authors,\theyear=year,\theinsight=insight,\thetag=tag}{%
\showpaperbasic{\thetitle}{\theauthors}{\theyear}{\theinsight}{\thetag}%
}
\end{literaturelist}
}

% 按指定标签显示带insight的简化版文献列表
\newcommand{\showbasiclistwithinsightbytag}[1]{%
\resetpapercount%
\begin{literaturelist}
\DTLforeach[\DTLisSubString{\thetag}{#1}]{literature} {\thetitle=title,\theauthors=authors,\theyear=year,\theinsight=insight,\thetag=tag}{%
\showpaperbasicwithinsight{\thetitle}{\theauthors}{\theyear}{\theinsight}{\thetag}%
}
\end{literaturelist}
}
 
% 需要添加的包
\RequirePackage{xstring}

% 智能标签解析与显示系统
\makeatletter
\newcommand{\showavailabletags}{%
  \textbf{Available Tags:}
  \def\@alltagslist{}%
  \DTLforeach{literature}{\thetag=tag}{%
      \ifx\thetag\empty
      \else
          \@processtags{\thetag}%
      \fi
  }%
  \@showtaglist%
  \\
}

% 处理逗号分隔标签的函数
\def\@processtags#1{%
  \StrSubstitute{#1}{ }{}[\@cleanedtags]%
  \@for\@singletag:=\@cleanedtags\do{%
      \ifx\@singletag\empty
      \else
          \@addtolistifnotexist{\@singletag}%
      \fi
  }%
}

% 添加标签到列表函数
\def\@addtolistifnotexist#1{%
  \@tempswafalse
  \ifx\@alltagslist\empty
      \edef\@alltagslist{#1}%
  \else
      \@checktagexists{#1}%
      \if@tempswa\else
          \edef\@alltagslist{\@alltagslist,#1}%
      \fi
  \fi
}

% 检查标签是否存在函数
\def\@checktagexists#1{%
  \@tempswafalse
  \@for\@existingtag:=\@alltagslist\do{%
      \StrCompare{#1}{\@existingtag}[\@result]%
      \ifnum\@result=0
          \@tempswatrue
      \fi
  }%
}

% 显示标签列表函数
\def\@showtaglist{%
  \ifx\@alltagslist\empty
      \textit{No tags found}
  \else
      \@for\@singletag:=\@alltagslist\do{%
          \ifx\@singletag\empty\else
              \displaytag{\@singletag}\space%
          \fi
      }%
  \fi
}
\makeatother

% ========================================================================
% 9. 统计和查询功能(支持标签统计)
% ========================================================================

% 获取文献总数
\newcommand{\getpapercount}{%
\DTLrowcount{literature}%
}

% 获取指定标签的文献数量
\newcommand{\getpapercountbytag}[1]{%
\DTLcondall{literature}{\DTLiseq{\thetag}{#1}}{\thetag=tag}%
}

% 显示统计信息
\newcommand{\showstats}{%
\begin{itemize}
\item Total papers: \getpapercount
\item Database: literature
\item Available fields: title, authors, year, insight, tag
\item \showavailabletags
\end{itemize}
}

% 按年份范围筛选显示(框架)
\newcommand{\showyearrange}[2]{%
\textbf{Papers from #1 to #2:}
}

% 按标签和年份范围组合筛选
\newcommand{\showbytagyearrange}[3]{%
\textbf{Papers with tag "#1" from #2 to #3:}
}

% ========================================================================
% 10. 实用工具命令
% ========================================================================

% 清空数据库
\newcommand{\clearliterature}{%
\DTLcleardb{literature}%
}

% 保存数据库到文件(接口定义)
\newcommand{\saveliterature}[1]{%
\PackageWarning{literature-system}{Save function not implemented yet}%
}

% 从文件加载数据库
\newcommand{\loadliterature}[1]{%
\PackageWarning{literature-system}{Load function not implemented yet}%
}

% ========================================================================
% 系统信息
% ========================================================================

\newcommand{\systeminfo}{%
\begin{center}
\fbox{\begin{minipage}{0.8\textwidth}
\textbf{Literature Auto-Sort System v2.0 (With Tags)}\\
\textit{Powered by datatool package}\\
Features: Auto-sorting, Multi-format display, Statistics, Tag filtering, Empty insight as $\emptyset$\\
File: literature-system.tex (Tag-enabled Version)
\end{minipage}}
\end{center}
}

% ========================================================================
% 文件结束标记
% ========================================================================
\typeout{Literature Auto-Sort System (Tag-enabled Version) loaded successfully!}
%
% 【系统架构】
% 1. 数据存储层:基于datatool包的数据库系统
% 2. 数据处理层:排序、筛选、统计功能
% 3. 显示控制层:多种格式化显示模板
% 4. 用户接口层:简化的命令接口
%
% 【技术依赖】
% - datatool: 核心数据处理包
% - longtable: 长表格支持(跨页显示)
% - array: 表格格式增强
% - xcolor: 颜色支持(标签高亮)
% - xstring: 字符串处理(标签解析)
%
% ========================================================================

% 必需的包(如果主文档没有加载的话)
\RequirePackage{datatool}
\RequirePackage{longtable}
\RequirePackage{array}
\RequirePackage{xcolor}

% ========================================================================
% 1. 数据库创建
% ========================================================================

% 创建文献数据库
\DTLnewdb{literature}

% ========================================================================
% 2. 数据添加接口(支持标签)
% ========================================================================

% 添加文献数据的主要命令(支持标签参数)
% 用法:\addliterature{标题}{作者}{年份}{洞察}{标签}
\newcommand{\addliterature}[5]{%
\DTLnewrow{literature}%
\DTLnewdbentry{literature}{title}{#1}%
\DTLnewdbentry{literature}{authors}{#2}%
\DTLnewdbentry{literature}{year}{#3}%
\DTLnewdbentry{literature}{insight}{#4}%
\DTLnewdbentry{literature}{tag}{#5}%
}

% 兼容旧版本的四参数命令(自动设置空标签)
\newcommand{\addliteratureold}[4]{%
\addliterature{#1}{#2}{#3}{#4}{}%
}

% ========================================================================
% 3. 排序功能
% ========================================================================

% 按年份排序(数值排序,从早到晚)
\newcommand{\sortbyyear}{%
\DTLsort{year}{literature}%
}

% 按标题排序(字母排序)
\newcommand{\sortbytitle}{%
\DTLsort{title}{literature}%
}

% 按作者排序(字母排序)
\newcommand{\sortbyauthors}{%
\DTLsort{authors}{literature}%
}

% 按标签排序(字母排序)
\newcommand{\sortbytag}{%
\DTLsort{tag}{literature}%
}

% 自定义排序命令
\newcommand{\sortby}[1]{%
\DTLsort{#1}{literature}%
}

% 复合排序(先按年份,再按标题)
\newcommand{\sortbyyeartitle}{%
\DTLsort{year,title}{literature}%
}

% 复合排序(先按标签,再按年份)
\newcommand{\sortbytagyear}{%
\DTLsort{tag,year}{literature}%
}

% ========================================================================
% 4. 空字段处理函数
% ========================================================================

% 处理空insight字段的显示函数
\makeatletter
\newcommand{\displayinsight}[1]{%
\def\@tempa{#1}%
\ifx\@tempa\@empty
\emptyset
\else
#1%
\fi
}

% 处理空tag字段的显示函数
\newcommand{\displaytag}[1]{%
\def\@tempa{#1}%
\ifx\@tempa\@empty
\textit{无标签}
\else
\textcolor{blue}{\texttt{#1}}%
\fi
}
\makeatother

% ========================================================================
% 5. 显示格式定义(支持标签显示)
% ========================================================================

% 文献编号计数器
\newcounter{papercount}

% 重置文献编号
\newcommand{\resetpapercount}{%
\setcounter{papercount}{0}%
}

% 单个文献条目显示命令(详细格式)
\newcommand{\showpaperentry}[5]{%
\stepcounter{papercount}%
\textbf{Paper \arabic{papercount}} & \\
\hline
Title & #1 \\
\hline
Authors & #2 \\
\hline
Year & #3 \\
\hline
Tag & \displaytag{#5} \\
\hline
Insight & \displayinsight{#4} \\
\hline\hline
}

% 简化版文献条目显示(只显示基本信息和标签)
\newcommand{\showpaperbasic}[5]{%
\stepcounter{papercount}%
\textbf{\arabic{papercount}.} & \textbf{#1} (#3) [\displaytag{#5}] \\
& \textit{#2} \\
\hline
}

% 带insight的简化版显示(包含标签)
\newcommand{\showpaperbasicwithinsight}[5]{%
\stepcounter{papercount}%
\textbf{\arabic{papercount}.} & \textbf{#1} (#3) [\displaytag{#5}] \\
& \textit{#2} \\
& \displayinsight{#4} \\
\hline
}

% ========================================================================
% 6. 显示环境定义
% ========================================================================

% 完整文献综述表格环境
\newenvironment{literaturereview}
{\begin{longtable}{|p{0.15\textwidth}|p{0.8\textwidth}|}
\hline
\textbf{Field} & \textbf{Content} \\
\hline
\endhead}
{\end{longtable}}

% 简化版文献列表环境
\newenvironment{literaturelist}
{\begin{longtable}{|p{0.05\textwidth}|p{0.9\textwidth}|}
\hline
\textbf{No.} & \textbf{Literature} \\
\hline
\endhead}
{\end{longtable}}

% ========================================================================
% 7. 批量显示命令(支持标签)
% ========================================================================

% 显示完整文献综述
\newcommand{\showfullreview}{%
\resetpapercount%
\begin{literaturereview}
\DTLforeach{literature}{\thetitle=title,\theauthors=authors,\theyear=year,\theinsight=insight,\thetag=tag}{%
\showpaperentry{\thetitle}{\theauthors}{\theyear}{\theinsight}{\thetag}%
}
\end{literaturereview}
}

% 显示简化版文献列表
\newcommand{\showbasiclist}{%
\resetpapercount%
\begin{literaturelist}
\DTLforeach{literature}{\thetitle=title,\theauthors=authors,\theyear=year,\theinsight=insight,\thetag=tag}{%
\showpaperbasic{\thetitle}{\theauthors}{\theyear}{\theinsight}{\thetag}%
}
\end{literaturelist}
}

% 显示带insight的简化版文献列表
\newcommand{\showbasiclistwithinsight}{%
\resetpapercount%
\begin{literaturelist}
\DTLforeach{literature}{\thetitle=title,\theauthors=authors,\theyear=year,\theinsight=insight,\thetag=tag}{%
\showpaperbasicwithinsight{\thetitle}{\theauthors}{\theyear}{\theinsight}{\thetag}%
}
\end{literaturelist}
}

% ========================================================================
% 8. 标签筛选显示功能
% ========================================================================

% 按指定标签显示完整文献综述
\newcommand{\showfullreviewbytag}[1]{%
\resetpapercount%
\begin{literaturereview}
\DTLforeach[\DTLisSubString{\thetag}{#1}]{literature} {\thetitle=title,\theauthors=authors,\theyear=year,\theinsight=insight,\thetag=tag}{%
\showpaperentry{\thetitle}{\theauthors}{\theyear}{\theinsight}{\thetag}%
}
\end{literaturereview}
}

% 按指定标签显示简化版文献列表
\newcommand{\showbasiclistbytag}[1]{%
\resetpapercount%
\begin{literaturelist}
\DTLforeach[\DTLisSubString{\thetag}{#1}]{literature} {\thetitle=title,\theauthors=authors,\theyear=year,\theinsight=insight,\thetag=tag}{%
\showpaperbasic{\thetitle}{\theauthors}{\theyear}{\theinsight}{\thetag}%
}
\end{literaturelist}
}

% 按指定标签显示带insight的简化版文献列表
\newcommand{\showbasiclistwithinsightbytag}[1]{%
\resetpapercount%
\begin{literaturelist}
\DTLforeach[\DTLisSubString{\thetag}{#1}]{literature} {\thetitle=title,\theauthors=authors,\theyear=year,\theinsight=insight,\thetag=tag}{%
\showpaperbasicwithinsight{\thetitle}{\theauthors}{\theyear}{\theinsight}{\thetag}%
}
\end{literaturelist}
}
 
% 需要添加的包
\RequirePackage{xstring}

% 智能标签解析与显示系统
\makeatletter
\newcommand{\showavailabletags}{%
  \textbf{Available Tags:}
  \def\@alltagslist{}%
  \DTLforeach{literature}{\thetag=tag}{%
      \ifx\thetag\empty
      \else
          \@processtags{\thetag}%
      \fi
  }%
  \@showtaglist%
  \\
}

% 处理逗号分隔标签的函数
\def\@processtags#1{%
  \StrSubstitute{#1}{ }{}[\@cleanedtags]%
  \@for\@singletag:=\@cleanedtags\do{%
      \ifx\@singletag\empty
      \else
          \@addtolistifnotexist{\@singletag}%
      \fi
  }%
}

% 添加标签到列表函数
\def\@addtolistifnotexist#1{%
  \@tempswafalse
  \ifx\@alltagslist\empty
      \edef\@alltagslist{#1}%
  \else
      \@checktagexists{#1}%
      \if@tempswa\else
          \edef\@alltagslist{\@alltagslist,#1}%
      \fi
  \fi
}

% 检查标签是否存在函数
\def\@checktagexists#1{%
  \@tempswafalse
  \@for\@existingtag:=\@alltagslist\do{%
      \StrCompare{#1}{\@existingtag}[\@result]%
      \ifnum\@result=0
          \@tempswatrue
      \fi
  }%
}

% 显示标签列表函数
\def\@showtaglist{%
  \ifx\@alltagslist\empty
      \textit{No tags found}
  \else
      \@for\@singletag:=\@alltagslist\do{%
          \ifx\@singletag\empty\else
              \displaytag{\@singletag}\space%
          \fi
      }%
  \fi
}
\makeatother

% ========================================================================
% 9. 统计和查询功能(支持标签统计)
% ========================================================================

% 获取文献总数
\newcommand{\getpapercount}{%
\DTLrowcount{literature}%
}

% 获取指定标签的文献数量
\newcommand{\getpapercountbytag}[1]{%
\DTLcondall{literature}{\DTLiseq{\thetag}{#1}}{\thetag=tag}%
}

% 显示统计信息
\newcommand{\showstats}{%
\begin{itemize}
\item Total papers: \getpapercount
\item Database: literature
\item Available fields: title, authors, year, insight, tag
\item \showavailabletags
\end{itemize}
}

% 按年份范围筛选显示(框架)
\newcommand{\showyearrange}[2]{%
\textbf{Papers from #1 to #2:}
}

% 按标签和年份范围组合筛选
\newcommand{\showbytagyearrange}[3]{%
\textbf{Papers with tag "#1" from #2 to #3:}
}

% ========================================================================
% 10. 实用工具命令
% ========================================================================

% 清空数据库
\newcommand{\clearliterature}{%
\DTLcleardb{literature}%
}

% 保存数据库到文件(接口定义)
\newcommand{\saveliterature}[1]{%
\PackageWarning{literature-system}{Save function not implemented yet}%
}

% 从文件加载数据库
\newcommand{\loadliterature}[1]{%
\PackageWarning{literature-system}{Load function not implemented yet}%
}

% ========================================================================
% 系统信息
% ========================================================================

\newcommand{\systeminfo}{%
\begin{center}
\fbox{\begin{minipage}{0.8\textwidth}
\textbf{Literature Auto-Sort System v2.0 (With Tags)}\\
\textit{Powered by datatool package}\\
Features: Auto-sorting, Multi-format display, Statistics, Tag filtering, Empty insight as $\emptyset$\\
File: literature-system.tex (Tag-enabled Version)
\end{minipage}}
\end{center}
}

% ========================================================================
% 文件结束标记
% ========================================================================
\typeout{Literature Auto-Sort System (Tag-enabled Version) loaded successfully!}
%
% 【系统架构】
% 1. 数据存储层:基于datatool包的数据库系统
% 2. 数据处理层:排序、筛选、统计功能
% 3. 显示控制层:多种格式化显示模板
% 4. 用户接口层:简化的命令接口
%
% 【技术依赖】
% - datatool: 核心数据处理包
% - longtable: 长表格支持(跨页显示)
% - array: 表格格式增强
% - xcolor: 颜色支持(标签高亮)
% - xstring: 字符串处理(标签解析)
%
% ========================================================================

% 必需的包(如果主文档没有加载的话)
\RequirePackage{datatool}
\RequirePackage{longtable}
\RequirePackage{array}
\RequirePackage{xcolor}

% ========================================================================
% 1. 数据库创建
% ========================================================================

% 创建文献数据库
\DTLnewdb{literature}

% ========================================================================
% 2. 数据添加接口(支持标签)
% ========================================================================

% 添加文献数据的主要命令(支持标签参数)
% 用法:\addliterature{标题}{作者}{年份}{洞察}{标签}
\newcommand{\addliterature}[5]{%
\DTLnewrow{literature}%
\DTLnewdbentry{literature}{title}{#1}%
\DTLnewdbentry{literature}{authors}{#2}%
\DTLnewdbentry{literature}{year}{#3}%
\DTLnewdbentry{literature}{insight}{#4}%
\DTLnewdbentry{literature}{tag}{#5}%
}

% 兼容旧版本的四参数命令(自动设置空标签)
\newcommand{\addliteratureold}[4]{%
\addliterature{#1}{#2}{#3}{#4}{}%
}

% ========================================================================
% 3. 排序功能
% ========================================================================

% 按年份排序(数值排序,从早到晚)
\newcommand{\sortbyyear}{%
\DTLsort{year}{literature}%
}

% 按标题排序(字母排序)
\newcommand{\sortbytitle}{%
\DTLsort{title}{literature}%
}

% 按作者排序(字母排序)
\newcommand{\sortbyauthors}{%
\DTLsort{authors}{literature}%
}

% 按标签排序(字母排序)
\newcommand{\sortbytag}{%
\DTLsort{tag}{literature}%
}

% 自定义排序命令
\newcommand{\sortby}[1]{%
\DTLsort{#1}{literature}%
}

% 复合排序(先按年份,再按标题)
\newcommand{\sortbyyeartitle}{%
\DTLsort{year,title}{literature}%
}

% 复合排序(先按标签,再按年份)
\newcommand{\sortbytagyear}{%
\DTLsort{tag,year}{literature}%
}

% ========================================================================
% 4. 空字段处理函数
% ========================================================================

% 处理空insight字段的显示函数
\makeatletter
\newcommand{\displayinsight}[1]{%
\def\@tempa{#1}%
\ifx\@tempa\@empty
\emptyset
\else
#1%
\fi
}

% 处理空tag字段的显示函数
\newcommand{\displaytag}[1]{%
\def\@tempa{#1}%
\ifx\@tempa\@empty
\textit{无标签}
\else
\textcolor{blue}{\texttt{#1}}%
\fi
}
\makeatother

% ========================================================================
% 5. 显示格式定义(支持标签显示)
% ========================================================================

% 文献编号计数器
\newcounter{papercount}

% 重置文献编号
\newcommand{\resetpapercount}{%
\setcounter{papercount}{0}%
}

% 单个文献条目显示命令(详细格式)
\newcommand{\showpaperentry}[5]{%
\stepcounter{papercount}%
\textbf{Paper \arabic{papercount}} & \\
\hline
Title & #1 \\
\hline
Authors & #2 \\
\hline
Year & #3 \\
\hline
Tag & \displaytag{#5} \\
\hline
Insight & \displayinsight{#4} \\
\hline\hline
}

% 简化版文献条目显示(只显示基本信息和标签)
\newcommand{\showpaperbasic}[5]{%
\stepcounter{papercount}%
\textbf{\arabic{papercount}.} & \textbf{#1} (#3) [\displaytag{#5}] \\
& \textit{#2} \\
\hline
}

% 带insight的简化版显示(包含标签)
\newcommand{\showpaperbasicwithinsight}[5]{%
\stepcounter{papercount}%
\textbf{\arabic{papercount}.} & \textbf{#1} (#3) [\displaytag{#5}] \\
& \textit{#2} \\
& \displayinsight{#4} \\
\hline
}

% ========================================================================
% 6. 显示环境定义
% ========================================================================

% 完整文献综述表格环境
\newenvironment{literaturereview}
{\begin{longtable}{|p{0.15\textwidth}|p{0.8\textwidth}|}
\hline
\textbf{Field} & \textbf{Content} \\
\hline
\endhead}
{\end{longtable}}

% 简化版文献列表环境
\newenvironment{literaturelist}
{\begin{longtable}{|p{0.05\textwidth}|p{0.9\textwidth}|}
\hline
\textbf{No.} & \textbf{Literature} \\
\hline
\endhead}
{\end{longtable}}

% ========================================================================
% 7. 批量显示命令(支持标签)
% ========================================================================

% 显示完整文献综述
\newcommand{\showfullreview}{%
\resetpapercount%
\begin{literaturereview}
\DTLforeach{literature}{\thetitle=title,\theauthors=authors,\theyear=year,\theinsight=insight,\thetag=tag}{%
\showpaperentry{\thetitle}{\theauthors}{\theyear}{\theinsight}{\thetag}%
}
\end{literaturereview}
}

% 显示简化版文献列表
\newcommand{\showbasiclist}{%
\resetpapercount%
\begin{literaturelist}
\DTLforeach{literature}{\thetitle=title,\theauthors=authors,\theyear=year,\theinsight=insight,\thetag=tag}{%
\showpaperbasic{\thetitle}{\theauthors}{\theyear}{\theinsight}{\thetag}%
}
\end{literaturelist}
}

% 显示带insight的简化版文献列表
\newcommand{\showbasiclistwithinsight}{%
\resetpapercount%
\begin{literaturelist}
\DTLforeach{literature}{\thetitle=title,\theauthors=authors,\theyear=year,\theinsight=insight,\thetag=tag}{%
\showpaperbasicwithinsight{\thetitle}{\theauthors}{\theyear}{\theinsight}{\thetag}%
}
\end{literaturelist}
}

% ========================================================================
% 8. 标签筛选显示功能
% ========================================================================

% 按指定标签显示完整文献综述
\newcommand{\showfullreviewbytag}[1]{%
\resetpapercount%
\begin{literaturereview}
\DTLforeach[\DTLisSubString{\thetag}{#1}]{literature} {\thetitle=title,\theauthors=authors,\theyear=year,\theinsight=insight,\thetag=tag}{%
\showpaperentry{\thetitle}{\theauthors}{\theyear}{\theinsight}{\thetag}%
}
\end{literaturereview}
}

% 按指定标签显示简化版文献列表
\newcommand{\showbasiclistbytag}[1]{%
\resetpapercount%
\begin{literaturelist}
\DTLforeach[\DTLisSubString{\thetag}{#1}]{literature} {\thetitle=title,\theauthors=authors,\theyear=year,\theinsight=insight,\thetag=tag}{%
\showpaperbasic{\thetitle}{\theauthors}{\theyear}{\theinsight}{\thetag}%
}
\end{literaturelist}
}

% 按指定标签显示带insight的简化版文献列表
\newcommand{\showbasiclistwithinsightbytag}[1]{%
\resetpapercount%
\begin{literaturelist}
\DTLforeach[\DTLisSubString{\thetag}{#1}]{literature} {\thetitle=title,\theauthors=authors,\theyear=year,\theinsight=insight,\thetag=tag}{%
\showpaperbasicwithinsight{\thetitle}{\theauthors}{\theyear}{\theinsight}{\thetag}%
}
\end{literaturelist}
}
 
% 需要添加的包
\RequirePackage{xstring}

% 智能标签解析与显示系统
\makeatletter
\newcommand{\showavailabletags}{%
  \textbf{Available Tags:}
  \def\@alltagslist{}%
  \DTLforeach{literature}{\thetag=tag}{%
      \ifx\thetag\empty
      \else
          \@processtags{\thetag}%
      \fi
  }%
  \@showtaglist%
  \\
}

% 处理逗号分隔标签的函数
\def\@processtags#1{%
  \StrSubstitute{#1}{ }{}[\@cleanedtags]%
  \@for\@singletag:=\@cleanedtags\do{%
      \ifx\@singletag\empty
      \else
          \@addtolistifnotexist{\@singletag}%
      \fi
  }%
}

% 添加标签到列表函数
\def\@addtolistifnotexist#1{%
  \@tempswafalse
  \ifx\@alltagslist\empty
      \edef\@alltagslist{#1}%
  \else
      \@checktagexists{#1}%
      \if@tempswa\else
          \edef\@alltagslist{\@alltagslist,#1}%
      \fi
  \fi
}

% 检查标签是否存在函数
\def\@checktagexists#1{%
  \@tempswafalse
  \@for\@existingtag:=\@alltagslist\do{%
      \StrCompare{#1}{\@existingtag}[\@result]%
      \ifnum\@result=0
          \@tempswatrue
      \fi
  }%
}

% 显示标签列表函数
\def\@showtaglist{%
  \ifx\@alltagslist\empty
      \textit{No tags found}
  \else
      \@for\@singletag:=\@alltagslist\do{%
          \ifx\@singletag\empty\else
              \displaytag{\@singletag}\space%
          \fi
      }%
  \fi
}
\makeatother

% ========================================================================
% 9. 统计和查询功能(支持标签统计)
% ========================================================================

% 获取文献总数
\newcommand{\getpapercount}{%
\DTLrowcount{literature}%
}

% 获取指定标签的文献数量
\newcommand{\getpapercountbytag}[1]{%
\DTLcondall{literature}{\DTLiseq{\thetag}{#1}}{\thetag=tag}%
}

% 显示统计信息
\newcommand{\showstats}{%
\begin{itemize}
\item Total papers: \getpapercount
\item Database: literature
\item Available fields: title, authors, year, insight, tag
\item \showavailabletags
\end{itemize}
}

% 按年份范围筛选显示(框架)
\newcommand{\showyearrange}[2]{%
\textbf{Papers from #1 to #2:}
}

% 按标签和年份范围组合筛选
\newcommand{\showbytagyearrange}[3]{%
\textbf{Papers with tag "#1" from #2 to #3:}
}

% ========================================================================
% 10. 实用工具命令
% ========================================================================

% 清空数据库
\newcommand{\clearliterature}{%
\DTLcleardb{literature}%
}

% 保存数据库到文件(接口定义)
\newcommand{\saveliterature}[1]{%
\PackageWarning{literature-system}{Save function not implemented yet}%
}

% 从文件加载数据库
\newcommand{\loadliterature}[1]{%
\PackageWarning{literature-system}{Load function not implemented yet}%
}

% ========================================================================
% 系统信息
% ========================================================================

\newcommand{\systeminfo}{%
\begin{center}
\fbox{\begin{minipage}{0.8\textwidth}
\textbf{Literature Auto-Sort System v2.0 (With Tags)}\\
\textit{Powered by datatool package}\\
Features: Auto-sorting, Multi-format display, Statistics, Tag filtering, Empty insight as $\emptyset$\\
File: literature-system.tex (Tag-enabled Version)
\end{minipage}}
\end{center}
}

% ========================================================================
% 文件结束标记
% ========================================================================
\typeout{Literature Auto-Sort System (Tag-enabled Version) loaded successfully!}